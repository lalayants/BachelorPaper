\documentclass[
candidate, % document type
subf, % use and configure subfig package for nested figure numbering
times, % use Times font as main
]{disser}

% Кодировка и язык
\usepackage[T2A, T1]{fontenc} % поддержка кириллицы
\usepackage[utf8]{inputenc} % кодировка исходного текста
\usepackage[english,russian]{babel} % переключение языков

% Геометрия страницы и графика
\usepackage[left=3cm, right=15mm, top=2cm, bottom=2cm]{geometry} % поля страницы
\setlength{\parindent}{1.25cm}
\usepackage{graphicx} % подключение графики
\usepackage{pdfpages} % вставка pdf-страниц
\usepackage{placeins}

% Таблицы
\usepackage{array} % расширенные возможности для работы с таблицами
\usepackage{tabularx} % автоматический подбор ширины столбцов
\usepackage{dcolumn} % выравнивание чисел по разделителю

% Математика
\usepackage{bm} % полужирное начертание для математических символов
\usepackage{amsmath} % дополнительные математические возможности
\usepackage{amssymb} % дополнительные математические символы

% Библиография и ссылки
\usepackage{cite} % поддержка цитирования
\usepackage{hyperref} % создание гиперссылок
% \usepackage[hidelinks, draft]{hyperref} % создание гиперссылок

% \usepackage{nohyperref} 

% Прочее
\usepackage{color} % работа с цветом
\usepackage{epstopdf} % конвертация eps в pdf
\usepackage{multirow} % объединение ячеек таблиц по вертикали
\usepackage{afterpage} % вставка материала после текущей страницы
\usepackage[font={normal}]{caption} % настройка подписей к рисункам и таблицам
\usepackage[onehalfspacing]{setspace} % полуторный интервал
\usepackage{fancyhdr} % установка колонтитулов
\usepackage{listings} % поддержка вставки исходного кода
\usepackage{booktabs}
\usepackage{caption}
\usepackage{setspace}
\DeclareCaptionLabelFormat{myfigure}{Рисунок~\thefigure\ --~}
\captionsetup[figure]{
  labelformat=myfigure,
  labelsep=none,
  justification=centering,
  singlelinecheck=false,
  format=plain,
  skip=0pt,
  font={stretch=1}
}
% Redefine the label format for tables: #2 is the table number.
\DeclareCaptionLabelFormat{mytable}{Таблица~#2\ --~}
\captionsetup[table]{
  labelformat=mytable,       % Use our custom format
  labelsep=none,
  justification=raggedright, % Left-align the caption
  singlelinecheck=false,     % Always use the provided justification
  skip=0pt,                   % No extra vertical space after the caption
  format=plain,
  font={stretch=1}
}

% Установка шрифта Times New Roman
\renewcommand{\rmdefault}{ftm}

% Создание нового типа столбца для выравнивания содержимого по центру
\newcommand{\PreserveBackslash}[1]{\let\temp=\\#1\let\\=\temp}
\newcolumntype{C}[1]{>{\PreserveBackslash\centering}p{#1}}

% Настройка стиля страницы
\pagestyle{fancy}      % Использование стиля "fancy" для оформления страниц
\fancyhf{}              % Очистка текущих значений колонтитулов
\fancyfoot[C]{\thepage} % Установка номера страницы в нижнем колонтитуле по центру
\renewcommand{\headrulewidth}{0pt} % Удаление разделительной линии в верхнем колонтитуле

% Настройка подписей к изображениям и таблицам
\captionsetup{format=hang,labelsep=period}

% Использование полужирного начертания для векторов
\let\vec=\mathbf

% Установка глубины оглавления
\setcounter{tocdepth}{2}

% Указание папки для поиска изображений
\graphicspath{{images/}}

% Установка стилей страницы и главы
\pagestyle{footcenter}
\chapterpagestyle{footcenter}
