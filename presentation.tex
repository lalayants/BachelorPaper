\documentclass{beamer} % Использование класса beamer для создания презентации
\usetheme{Boadilla} % Применение темы Boadilla

\usepackage[T2A]{fontenc} % Установка кодировки шрифта
\usepackage[utf8]{inputenc} % Установка кодировки исходного текста
\usepackage[english,russian]{babel} % Подключение локализации и переносов для русского и английского языков
\usepackage{amsmath,amssymb} % Подключение математических символов и формул
\usepackage{autonum} % Автоматическая нумерация формул только при наличии ссылок на них
\usepackage{wrapfig} % Подключение пакета для обтекания текстом рисунков и таблиц
\usepackage{array} % Подключение пакета для работы с таблицами

\newcommand{\PreserveBackslash}[1]{\let\temp=\\#1\let\\=\temp} % Команда для сохранения обратной косой черты в ячейках таблицы
\newcolumntype{C}[1]{>{\PreserveBackslash\centering}p{#1}} % Создание нового типа столбца с центрированным содержимым

\usepackage[labelformat=empty]{caption} % Подключение пакета для настройки подписей и удаление названия "Рисунок"
\graphicspath{{./images/}{./images/logo/}} % Указание папок, где искать изображения

\setbeamertemplate{navigation symbols}{} % Удаление навигационных символов на слайдах
\usepackage{ragged2e} % Подключение пакета для улучшенного выравнивания текста
\newcommand{\jj}{\righthyphenmin=20 \justifying} % Команда для активации выравнивания по ширине с установкой минимального количества символов в слове перед переносом



\begin{document}




\begin{frame}
  \begin{center}\tiny
    Федеральное государственное автономное образовательное учреждение высшего образования \\
    «Национальный исследовательский университет ИТМО»\\
    Факультет систем управления и робототехники
  \end{center}

  \begin{center}
    \includegraphics[width=0.15\linewidth]{emb}
  \end{center}

  \begin{center}\tiny
    \textbf{КВАЛИФИКАЦИОННАЯ РАБОТА МАГИСТРА ТЕХНИКА И ТЕХНОЛОГИИ \\ ПО НАПРАВЛЕНИЮ 14040000.62 «ТЕХНИЧЕСКАЯ ФИЗИКА»}\\
    \vspace{0.2cm}
    \textbf{НА ТЕМУ:}\\
    \vspace{0.1cm}
    \textbf{<<РОЛЬ ДАЛЬНОДЕЙСТВИЯ ПРИТЯЖЕНИЯ В \\ ДИФФУЗИИ И СПЕКТРАХ ВОЗБУЖДЕНИЙ ПРОСТЫХ ЖИДКОСТЕЙ>>}
  \end{center}

  \vspace{0.3cm}

  \begin{columns}
    \begin{column}{0.50\textwidth}
      \begin{center}\tiny
        Выполнил: \\
        \vspace{0.1cm}
        \textbf{Лалаянц Кирилл Артемович}\\
        336700
        студент группы R34352 \\
      \end{center}
    \end{column}

    \begin{column}{0.50\textwidth}
      \begin{center}\tiny
        Научный руководитель:\\
        \vspace{0.1cm}
        \textbf{Шаветов Сергей Васильевич} \\
        доцент, к.т.н., заместитель декана ФСУиР\\
        начальник ДНИР \\
      \end{center}
    \end{column}
  \end{columns}

  \vspace{0.5cm}
  \begin{center}\tiny
    Москва, $2022$
  \end{center}
\end{frame}




\begin{frame}{Актуальность}
  \footnotesize{

    \textbf{Актуальность:}

    Остаются открытыми такие важные вопросы, как влияние потенциала взаимодействия между частицами на температурную зависимость коэффициента диффузии и насколько важны корреляции между спектрами возбуждения частиц и транспортными свойствами, а также, какую роль играет дальнодействие притяжения в скорости нуклеации в переохлажденных системах

    \vspace{1cm}

    \textbf{Практическая значимость:}

    \begin{itemize}
    \item Численно рассчитана диффузия и спектры на жидкостной бинодали веществ
    \item Разработан новый метод распознавания фаз, который универсален по отношению к размерности системы и формам кластеров, с сохранением точности расчета на уровне других методов
    \end{itemize}
  }
\end{frame}




\begin{frame}{Цель и задачи работы}
  \footnotesize{

    \textbf{Цель работы} -- установить связь дальнодействия притяжения потенциала взаимодействия и спектров возбуждений с транспортными свойствами жидкостей, а также выявить влияние дальнодействия притяжения на скорость нуклеации.

    \vspace{0.5cm}

    \textbf{Задачами работы являются:}
    \begin{itemize}
    \item Расчет фазовых диаграмм для 2D и 3D систем частиц, взаимодействующих посредством обобщенного потенциала Леннарда-Джонса с различными степенями притяжения.
    \item Адаптация метода кластеризации данных DBSCAN для изучения молекулярных систем и его сравнение с другими методами.
    \item Расчет и анализ транспортных свойств и коллективных возбуждений на жидкостных бинодалях.
    \item Применение нового метода распознавания фаз для изучения скорости нуклеации в переохлажденных системах Леннарда-Джонса с различным дальнодействием притяжения.
    \end{itemize}
  }
\end{frame}


\begin{frame}
  \centering \Huge \textcolor{blue}{Спасибо за внимание!}
\end{frame}

\end{document}
