Алгоритм OC-SORT \cite{cao2023observation} был представлен чуть позже ByteTrack в 2022 году. 
В своей работе авторы работают над проблемой интерполяции траекторий. Более ранние подходы использовали только фильтр Калмана и опирались исключительно на его оценки в случае перекрытия или потери объекта. 
Это приводит к очевидным минусам таким как: 
\begin{itemize}
    \item[--] накопление шумов. Обычно скорость перемещения объектов на изображении всего несколько пикселей за кадр, а потому от кадра к кадру она может меняться в разы;
    \item[--] в случае нелинейного движения линейная аппроксимация также приводит к большим ошибкам на продолжительных отрезках времени.
\end{itemize}
Эти недостатки не критичны в том случае, когда потеря случается на короткий промежуток времени. Однако, если использовать такую оценку на протяжении десятка кадров без обновления фильтра Калмана, то накопленная ошибка будет уже значительна и не позволит сопоставить вновь найденный объект с его изначальной траекторией.
Авторы предлагают избежать этих проблем через смену подхода оценки на опирающийся больше на наблюдения. 

Первая из них решается благодаря ORU (англ. Observation-Centric Re-Update -- переобновление на основе наблюдений). После удачного сопоставления траектории с объектом после периода потери из поля зрения предлагается откатить состояние фильтра до того, что было в момент потери. После этого фильтр обновляют, подавая ему на вход виртуальную траекторию от точки потери до точки нахождения. Благодаря этому удается нивелировать ошибку состояния фильтра, накопленную при его слепом обновлении во время потери.

Второй способ, использующийся в работе -- OCM (англ. Observation-Centric Momentum -- момент на основе наблюдений). На основе последних нескольких показаний (как правило 3), распределенных через равные, не слишком маленькие промежутки времени, строится предположение траектории движения. После этого находится точка, которая меньше всего от него отклоняется. Этот способ лучше традиционного использования последних двух точек и линейной экстраполяции, так как меньше подвержен шумам. 

Полученный подход работает чуть лучше, чем ByteTrack: MOTA -- 78.0; IDF1 -- 77.5; HOTA -- 63.2.