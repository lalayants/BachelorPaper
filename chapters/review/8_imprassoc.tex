Алгоритм ImprAssOC \cite{stadler2023improved} также был представлен в 2023 году. 

Более ранние алгоритмы, как это было упомянуто в описании метода ByteTrack, делили объекты на две группы: с низкой и высокой степенью уверенности. 
Из-за этого объекты из группы с низкой степенью уверенности не могли быть сопоставлены с треками, объект которых потерян, а также в целом приоритет всегда отдавался объектам с высокой. 
Авторы избегают этого, составляя матрицу расстояний сразу для всех объектов. Расстояния до объектов из группы с низкой степенью уверенности после этого нормализуются, чтобы быть одной размерности. 

Более того, авторы улучшают алгоритм ассоциации, потому что посчитали, что используемое в BoT-SORT взятие минимума из IoU и ReID не использует все доступные данные, а StrongSORT использует расстояние Махаланобиса, которое по их мнению дает только грубую оценку при высокой степени неизвестности.
Для этого они комбинируют подходы и берут взвешенную сумму визуальных особенностей и расстояния до траектории, в случае если IoU больше какого-то значения, иначе просто присваивается максимальная степень непохожести.

Также авторы вводят новый подход для инициализации траектории. Раньше это либо происходило сразу для любого найденного объекта, что провоцировало создание лишних траекторий для FP результатов детектирования, либо инициализировали траекторию, если объект был найден n раз подряд, что искусственно создавало кадры, на которых объект найден, но еще не отслеживается. 
Суть подхода заключается в том, чтобы откинуть все FP объекты. Для всех несопоставленных с траекторией объектов вычисляется IoU с сопоставленными. Если больше какого-то граничного значения, то результат детектирования признается дубликатом и отбрасывается. 

Для обучения сети детектора используется MOT17, MOT20, CrowdHuman, ETH и CityPersons \cite{zhang2017citypersons}. Получены следующие показатели: MOTA -- 82.2; IDF1 -- 82.1; HOTA -- 66.4.