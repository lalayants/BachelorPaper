Метрика МОТА\cite{bernardin2008evaluating} (англ. Multiple Object Tracking Accuracy -- точность отслеживания нескольких объектов) была предложена еще в 2008 году и является старейшей из всех используемых в работе. 
Ее авторы ставили целью создать критерий, который позволит оценивать точность выполнения задачи слежения. 
Подчеркивается важность способности метрики учитывать постоянство идентификации, точность оценки положения объекта и итоговой траектории.
Более того, был сделан акцент на минимизации параметров, а также интуитивной понятности для человека и применимости для задач слежения в любой постановке.

В итоге, был предложен следующий способ вычисления 
\begin{equation}
    \text{MOTA} = 1 - \frac{\sum_{t=0}^{T}(\text{FP} + \text{FN} + \text{IDSW})}{\sum_{t=0}^{T}g_t},
    \label{eq:mota}
\end{equation}
где FP -- количество неверно найденных объектов, которые на самом деле не были отмечены на изображении при разметке; FN -- количество объектов, которые были отмечены при разметке, но не были найдены алгоритмом на изображении; IDSW -- сколько объектов найденных ранее не были правильно идентифицированы и получили новый уникальный идентификационный номер, \(g_t\) -- общее количество объектов, отмеченных при разметке на кадре.
Из формулы \ref{eq:mota} видно -- чем больше показатель метрики, тем лучше справляется алгоритм.

Существует критика метрика MOTА, связанная с суммированием ошибок разного рода без какой-либо их предобработки \cite{dendorfer2021motchallenge}, а так же плохой репрезентацией при отслеживании с использованием нескольких камер \cite{ristani2016performance}. Тем не менее, MOTA используется практически в каждом исследовании по теме и на практике доказала свою состоятельность в оценивании качества отслеживания объектов на видеоизображениях. Именно поэтому было принято решение включить ее в качестве критерия для сравнения методов в данной работе.