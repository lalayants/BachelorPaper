Алгоритм StrongSORT \cite{du2023strongsort} был представлен в 2023 году. 
Авторы поднимают вопрос отсутствия стандарта тестирования различных алгоритмов, в связи с чем иногда преимущество получают не те, 
которые работают эффективнее, а для которых был лучше обучен детектор.
Это затрудняет процесс объективного сравнения современных подходом между собой. 
Так же авторы утверждают, что если использовать старый алгоритм DeepSORT \cite{wojke2017deepsort}, но дооснастить его современными решениями, то получится конкурентноспособный подход. Свое желание актуализировать DeepSORT они мотивируют его легковесностью и эффективностью.

В итоге к алгоритму DeepSORT были добавлены:
\begin{itemize}
    \item[--] улучшенный экстрактор особенностей объектов;
    \item[--] компенсация движения камеры;
    \item[--] модифицированный для лучшей работы с шумами фильтр Калмана;
    \item[--] обновление особенностей объекта с помощью скользящего экспоненциального среднего;
    \item[--] реидентификация не только по внешнему виду, но и расстоянию Махаланобиса.
    % \item[--] 
\end{itemize}
Более того, авторы добавляют два новых алгоритма AFLink и GSI, которые позволяют качественнее обрабатывать моменты, когда объект теряется из кадра или перекрывается другими. 

Работа авторов демонстрирует эффективность модульного подхода в задаче отслеживания объектов на изображении. Алгоритм StrongSORT был получен наслаиванием друг на друга различных подходов, а так же добавлением нескольких новых методик.
Каждая из модификация дает прирост производительности в несколько пунктов, что в общей сумме повышает эффективность алгоритма и позволяет ему обойти все современные по показателям метрик: MOTA -- 77.1; IDF1 -- 82.3; HOTA -- 69.6.
