Метрика IDF1 \cite{ristani2016performance} рекомендована для анализа производительности трекера в качестве дополнительной в связи с некоторым недостатками MOTA.

Авторы считали, что основная проблема других оценок -- арифметические операции с разнородными ошибками, которые могут оказывать влияние на интерпретируемость полученного результата. В общем, они оценивали вероятность появления ошибок идентификации и детектирования в каждый момент времени, а не точность итоговых траекторий у каждого объекта. 

Предложенная метрика IDF1 фокусировалась на оценке того, насколько качественно и долго трекер сохраняет идентичность объектов. Для этого, полученные с помощью трекера объекты, их идентичности и итоговые траектории сопоставляются с размеченными на основе методов графовой оптимизации, благодаря чему размеченные траектории сопоставляются с полученными на основе минимизации ошибки. Получившаяся оценка является более репрезентативной с точки зрения отслеживания конкретного объекта, так как оценивает именно задачу сохранения идентификации и точности итоговых траекторий.  
При этом у метрики есть и свои недостатки, поэтому авторы позиционирует ее как дополнительную для более комплексной оценки. 

Метрики IDF1 представляет собой
\begin{equation}
    \label{eq:idf1}
    \text{IDF1} = \frac{2 \text{IDTP}}{2\text{IDTP} + \text{IDFP} + \text{IDFN}},
\end{equation}
где IDTP -- количество верно идентификационных отслеживаемых объектов; IDFN -- количество ненайденных объектов, которые были в разметке, но которым не были сопоставлены траектории; IDFP -- количество ложных срабатываний, когда был найден несуществующий объект. 
Расчет всех этих значений происходит после упомянутого ранее процесса сопоставления полученных траекторий с размеченными на основе оптимизации графа. 

% Полученная оценка является ед

В итоге, благодаря оценке с точки зрения качества сохранения присвоенных ранее идентификаций, метрика IDF1 является отличным дополнением для MOTA и была выбрана для учета в сравнении методов отслеживания объектов на видеоизображениях в рамках данной работы.