Алгоритм Deep OC-SORT \cite{maggiolino2023deep} последний из рассматриваемых в этой работе. 
Авторы решили совместить Deep-SORT и OC-Sort. Это чем-то похоже на StrongSORT, но со своими особенностями. 

Во-первых, предложен улучшенный алгоритм компенсации движения камеры. Так, фильтра Калмана, OCM и ORU получают на вход данные с компенсацией погрешностей, вызванных ее движением. 

Во-вторых, предложен улучшенный алгоритм скользящего экспоненциального среднего. Авторы по показателю уверенности в результате детектирования адаптивно меняют параметр, отвечающий за быстроту усреднения. Мотивация этого кроется в том, что если степень уверенности низкая -- объект либо смазан, либо перекрыт другими объектами, а значит делать поправку на новые особенности стоит в меньшей степени.

В итоге, полученный подход работает относительно лучше, чем OC-Sort: MOTA -- 75.6; IDF1 -- 79.2; HOTA -- 63.9. Хоть МОТА немного и упала, но IDF1 и HOTA выросли, что свидетельствует о лучшей работе с реидентификацией и особенностями объектов.
