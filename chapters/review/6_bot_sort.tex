BoT-SORT \cite{aharon2022bot} третий алгоритм, представленный в 2022 году. 
В своей работе авторы пишут, что алгоритмы представленные раннее имеют множество недостатков.
Во-первых, фильтры Калмана используется для предсказания отношения ширины области интереса к ее высоте, что является причиной неточности оценки. 
Во-вторых, авторы утверждают, что в случае временной потери объекта из поля зрения реидентификация зачастую проваливается из-за неучтенного движения камеры. 
В-третьих, использование ReID моделей ведет к оптимизации метрики IDF1, в то время как IoU оптимизирует MOTA, поэтому существуют вероятность улучшения производительности по обеим метрикам через введение нового способа определения похожести на основе двух этих. 

Для решения первой проблемы было предложено изменить классический для задачи отслеживания объектов на изображении вектор состояния фильтра Калмана 
\begin{equation}
    \label{eq:bot_old_kalman}
    x = [x_c, y_c, a, h, \dot x_c, \dot y_c, \dot a, \dot h]^T,
\end{equation}
где \((x_c, y_c)\) -- координаты области интереса, \(a\) -- отношения сторон области интереса, а \(h\) -- ее высота.
Вместо этого предложен вектор состояния
\begin{equation}
    \label{eq:bot_new_kalman}
    x = [x_c, y_c, w, h, \dot x_c, \dot y_c, \dot w, \dot h]^T,
\end{equation}
где \(w\) -- ширина области интереса. Авторы не очень понимают, почему это улучшает результат, но метрики растут. 

Для решения второй они добавляют использование алгоритм CMC (англ. Camera Motion Compensation -- компенсация движения камеры). При допущении, что объекты от кадра к кадру перемещаются на изображении несильно, данный подход значительно повышает робастность сопоставления траекторий и найденных объектов при нестабильном положении камеры в пространстве. 

Последняя проблема решается через комбинацию матриц близости IoU и косинусовых расстояний близости ReID модели. Для этого просто берется минимум для каждой пары объектов. 

Применения всех перечисленных подходов позволило добиться лучшего результата среди всех трекеров на тот момент: MOTA -- 78.5; IDF1 -- 82.1; HOTA -- 69.2.