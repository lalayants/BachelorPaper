\newpage
\begin{flushleft}
  \textbf{\large 3 Разработка системы для проведения экспериментальных исследований }
\end{flushleft}
\refstepcounter{chapter}
\addcontentsline{toc}{chapter}{3 Разработка системы для проведения экспериментальных исследований}
В данной главе будут рассмотрены выбранные для системы апробации компоненты и описан процесс их подготовки.

\section{Выбор микрокомпьютера}
На данный момент на рынке представлено множество микрокомпьютеров. Наиболее распространенными из них являются Orange Pi, Nvidia Jetson и Raspberry Pi. 

Идеальным выбором для запуска тяжеловесных алгоритмов является Nvidia Jetson, но он обладает рядом недостатков:
\begin{itemize}
  \item высокая цена;
  \item высокое энергопотребление;
  \item относительно низкая распространенность в России;
  \item большой вес у старших моделей.
\end{itemize}

Перечисленные недостатки критичны, особенно в задачах мобильной робототехники, где энергоэффективность и рабочий вес наиболее чувствительные параметры. 
Orange Pi является подобием Raspberry Pi, но при этом обладает худшей поддержкой и меньшей распространенностью. 

Raspberry Pi является идеальным выбором в силу следующих причин:
\begin{itemize}
  \item низкая цена -- в России можно найти предложения за 10000 рублей;
  \item малый вес -- плата весит 50г;
  \item высокая энергоэффективность;
  \item доступность в России;
  \item отличная поддержка;
  \item большое количество информации в интернете;
  \item внешние вычислительные модули TPU, разработанные специально для платформы. 
\end{itemize}

Именно поэтому для стенда апробации был выбран микрокомпьютер Raspberry PI5 8gb.

\section{Выбор дополнительного вычислительного модуля}
В качестве дополнительного вычислительного модуля в работе будет использоваться TPU (англ. Tensor Processing Unit -- устройство обработки тензоров). Использование TPU позволяет оптимизировать работу с тензорами с помощью параллельных вычислений.
Благодаря этому, скорость работы нейронных сетей возрастает в разы. На данный момент на рынке РФ представлены TPU двух компаний: Google Coral и Hailo. 

Hailo является более новой разработкой, мощнее и эффективнее. Однако, в связи с его новизной, все еще остаются проблемы и баги, которые могут быть критичны при использовании. Так же сама процесс взаимодействия с Hailo требует переписывания всего кода, так как требует использования отдельной библиотеки. 

В свою очередь Google Coral требует только конвертировать веса нейронной сети в специальный формат, после чего все происходит в автоматическом режиме. Для его работы не требуется переписывать ни единой строчки кода. 
Так же существуют различные версии, которые позволяют подключение с использованием различных портов: M2, Mini PCIe, USB. 
Существует еще и версия с двумя ядрами, что дает возможность одновременно запускать две разных нейронных сети независимо друг от друга. 

По совокупности удобства использования, доступности и проверенной временем надежности выбор пал на внешний вычислительный модуль Google Coral M.2 Accelerator with Dual Edge TPU. 

\section{Настройка и подключение компонентов}
Для удобства работы была выбрана операционная система Raspberry PI5, так как она сразу включает в себя все нужные драйвера для Raspberry Pi, а также удобный интерфейс для их настройки. 
Дополнительно нужно только поставить драйвера Google Coral, представленные на официальном сайте Google Coral. 

\section{Выводы по главе}
В главе были разобраны представленные на рынке варианты микрокомпьютеров и вычислительных модулей TPU. В итоге выбраны комплектующие для стенда апробации:
\begin{itemize}
  \item микрокомпьютер -- Rаsрbеrry рi 5 8GВ.
  \item вычислительный модуль -- Google Coral TPU Dual;
  \item плата для подключения -- Рinеboards Ai Dual.
\end{itemize}
% TODO: CHECK WEIGHT 
Общая стоимость компонент составляет 28000 рублей; масса -- 100г. 