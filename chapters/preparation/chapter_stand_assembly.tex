\newpage
\begin{flushleft}
  \textbf{\large 3 Разработка системы апробации }
\end{flushleft}
\refstepcounter{chapter}
\addcontentsline{toc}{chapter}{3 Разработка системы апробации}
В данной главе будут рассмотрены выбранные для системы апробации компоненты и описан процесс их подготовки.

\section{Выбор микрокомпьютера}
На данный момент на рынке представлено множество микрокомпьютеров. Наиболее распространенными из них являются Orange Pi, Nvidia Jetson и Raspberry Pi. 

Идеальным выбором для запуска тяжеловесных алгоритмов является Nvidia Jetson, но он обладает рядом недостатков такими как:
\begin{itemize}
  \item высокая цена;
  \item высокое энергопотребление;
  \item относительно низкая распространенность в России;
  \item большой вес у старших моделей.
\end{itemize}
Перечисленные недостатки критичны, особенно в задачах мобильной робототехники, где энергоэффективность и рабочий вес наиболее чувствительные параметры. 
Orange Pi является подобием Raspberry Pi, но при этом обладает худшей поддержкой и меньшей распространенностью. 

Raspberry Pi является идеальным выбором в силу следующих причин:
\begin{itemize}
  \item низкая цена -- в России можно найти предложения за 10000 рублей;
  \item малый вес -- плата весит 50г без кулера и 100г с ним;
  \item высокая энергоэффективность;
  \item доступность в России;
  \item отличная поддержка;
  \item большое количество информации в интернете;
  \item внешние вычислительные модули TPU, разработанные специально для платформы. 
\end{itemize}

Именно поэтому для стенда апробации был выбран микрокомпьютер Raspberry PI5 8gb.

\section{Выбор дополнительного вычислительного модуля}
На данный момент на рынке РФ представлены TPU двух компаний: Google Coral и Hailo. 

Hailo является более новой разработкой, мощнее и эффективнее. Однако, в связи с его новизной, все еще остаются проблемы и баги, которые могут быть критичны при использовании. Так же сама процесс взаимодействия с Hailo требует переписывания всего кода, так как требует использования отдельной библиотеки. 

В свою очередь Google Coral требует только конвертировать веса нейронной сети в специальный формат, после чего все происходит в автоматическом режиме. Для его работы не требуется переписывать ни единой строчки кода. 
Так же существуют различные версии, которые позволяют подключение с использованием различных портов: M2, Mini PCIe, USB. 
Существует еще и версия с двумя ядрами, что дает возможность одновременно запускать две разных нейронных сети независимо друг от друга. 

По совокупности удобства использования, доступности и проверенной временем надежности выбор пал на внешний вычислительный модуль Google Coral M.2 Accelerator with Dual Edge TPU. 

\section{Настройка и подключение компонент}

\section{Выводы по главе}