% \newpage
% \begin{center}
%   \textbf{\large АННОТАЦИЯ}
% \end{center}


% Объектом исследования данной работы стали системы частиц, взаимодействующих посредством обобщенного потенциала Леннарда-Джонса с переменной степенью притяжения.
% С помощью данных систем выявлялось влияние дальнодействия притяжения на фазовые диаграммы, роль дальнодействия в транспортных свойствах, а также влияние на спектры возбуждений.
% Для решения поставленной задачи методом молекулярной динамики были смоделированы системы.
% Пост обработка проводилась с использованием MATLAB и python.

% В ходе работы были проведены моделирования систем с различным дальнодействием притягивающей ветви потенциала взаимодействия.
% Рассчитаны фазовые диаграммы систем, их транспортные свойства и спектры возбуждений на бинодали жидкость-газ.
% Предложен новый метод классификации частиц на конденсат, газ и поверхность в системах с фазовым расслоением.
% На основании полученных результатов, было впервые выявлено влияние дальнодействия притяжения в молекулярных системах на фазовые диаграммы, положения тройных и критических точек и корреляции транспортных свойств со спектрами возбуждений на бинодали жидкость-газ.
% С помощью нового метода классификации частиц, основанного на методе кластеризации DBSCAN (Density-Based Spatial Clustering of Applications with Noise), был разработан метод построения фазовых диаграмм и анализа нуклеации.

% \onehalfspacing
% \setcounter{page}{2}

\newpage
\renewcommand{\contentsname}{\centerline{\large СОДЕРЖАНИЕ}}
\tableofcontents
\setcounter{page}{2}

\newpage
\begin{flushleft}
  \textbf{\large ВВЕДЕНИЕ}
\end{flushleft}
\addcontentsline{toc}{chapter}{ВВЕДЕНИЕ}


\textbf{Актуальность}

За прошедшие годы в процессе диффузии в жидкостях был достигнут определённый прогресс~\cite{zhang2022bytetrack}, однако знания в данной области все еще довольно ограничены.
Существуют приблизительные соотношения, которые могут с разной степенью точности описать диффузию в различных системах.
Простейшей оценкой температурной зависимости коэффициента диффузии в жидкости является закон Аррениуса~\cite{10.1126/science.278.5336.257}.
Однако пренебрежение динамической вязкостью и особенностями реального потенциала взаимодействия делает данный закон непригодным для точного определения коэффициента диффузии в широком диапазоне температур.
Среди других соотношений диффузии в жидкостях, упомянем избыточное энтропийное масштабирование коэффициентов переноса~\cite{10.1103/physreva.15.2545, 10.1038/381137a0, 10.1063/1.5055064}, соотношение температуры замерзания и плотности~\cite{10.1103/physreve.62.7524, 10.1063/1.5022058, 10.1063/1.5044703, 10.1103/physreve.103.042122}, а также соотношение Стокс-Эйнштейна между коэффициентами вязкости диффузии и сдвига~\cite{10.1063/1.446338, 10.1002/BBPC.19900940313, 10.1103/physreve.95.052122, 10.1063/1.5080662, 10.1080/00268976.2019.1643045}.
Существуют методы, позволяющие достаточно точно прогнозировать коэфицент диффузии в конкретных системах, в том числе в широком интервале температур, вплоть до критической точки и в закритической области~\cite{10.1063/1.1607953, 10.1016/j.camwa.2019.11.012, 10.1063/1.441097}.
Получены обширные результаты численного моделирования~\cite{10.1063/1.1786579, 10.1016/j.fluid.2011.03.002}.
В настоящее время применяются методы машинного обучения~\cite{10.1063/5.0011512}.
Но остаются открытыми следующие важные вопросы:
\begin{enumerate}
\item Какое влияние имеет потенциал взаимодействия между частицами на температурную зависимость коэффициента диффузии;
\item Насколько важны корреляции между спектрами возбуждений и транспортными свойствами.
\end{enumerate}

\newpage
