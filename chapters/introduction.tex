\newpage
\renewcommand{\contentsname}{\centerline{\large СОДЕРЖАНИЕ}}
\tableofcontents
% \setcounter{page}{2}

\newpage
\begin{flushleft}
  \textbf{\large ВВЕДЕНИЕ}
\end{flushleft}
\addcontentsline{toc}{chapter}{ВВЕДЕНИЕ}

Темпы роста количества собираемой информации стабильно увеличиваются. Системы анализа данных производят обработку всего этого потока.
Они проводят миллионы вычислений, целью которых является выявление полезных закономерностей в повседневной жизни для повышения производительности труда и оптимизации расходов. 
Польза этого направления была признана руководителями из различных сфер: от компаний, ставящих целью максимизацию прибыли, до глав государственных институтов, улучшающих процессы взаимодействия миллионов людей.

Одним из наиболее бурно развивающихся направлений анализа данных является компьютерное зрение. 
За последние 20 лет прорыв в области глубоких нейронных сетей позволил решать задачи детектирования и классификации объектов на изображениях с принципиально недоступной до этого точностью.
На основе новых возможностей происходит активный поиск способов отслеживания перемещений объектов интереса. По этой теме ежегодно выходит множество научных статей, применимость которых очевидна как в робототехнике, так и в бизнесе, урбанистике, архитектуре, государственном правлении и обеспечении порядка.

Эпоха интернета вещей проходила свое становление параллельно с развитием технологий обработки информации.
Любой гаджет с подключением к интернету способен считывать сотни показаний ежедневно: начиная температурой воздуха и заканчивая поведенческими особенностями владельца.
Весь этот поток сырых данных идет на централизованные сервера для последующей обработки, так как для этого требуются большие вычислительные мощности.

У описанного выше подхода есть свои минусы. Во-первых, увеличивается нагрузка на сеть, ведь необработанные данные, как правило, имеют больший вес. Во-вторых, возникают риски потери, вызванных техническими сбоями со стороны сложно устроенного сервера или проблемами с соединением.
Перечисленные недостатки делают его вовсе не пригодным для некоторых задач, например, автономной робототехники, где непредвиденные задержки в сотые доли секунды могут привести к катастрофе. 

Данная работа ставит своей целью исследовать проблему требовательных вычислений на микрокомпьютер для задачи отслеживания объектов на видеоизображениях в городской среде. 
Исследование позволит оценить возможность проведения анализа сырых данных сразу в момент их получения и найти наиболее эффективные алгоритмы не только по метрикам качества, как это происходит в статьях, но и по скорости работы.

Для получения результата, соответствующего поставленной цели, составлен список задач, подлежащих решению в работе:
\begin{enumerate}
    \item провести аналитический обзор существующих алгоритмов;
    \item подготовить стенд для апробации;
    \item подготовить эксперимент для сравнения различных алгоритмов на основании метрик HOTA, MOTA, IDF1 и показателей производительности.;
    \item проанализировать полученные результаты.
\end{enumerate}

% Исходные данные к работе:
% \begin{enumerate}
%     \item В качестве низкопроизводительного устройства использовать микрокомпьютер Raspberry Pi 5 с внешним вычислительным модулем TPU.
%     \item Рассмотреть не менее 3 различных алгоритмов.
%     \item Для сравнения использовать набор данных MOT17.
% \end{enumerate}

% Выработанные рекомендации по 
\newpage
